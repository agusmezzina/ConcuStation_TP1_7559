% informe.tex - Documentación sobre el trabajo práctico para entregar

\documentclass{article}
\usepackage[utf8]{inputenc}
\usepackage[spanish]{babel}
\usepackage{graphicx}

\begin{document}

\title{Informe Trabajo Práctico 1}
\author{Victoria Perelló (89608)\\
        \and
        Agustín Mezzina (89637)\\
	\and
        Pablo Rodríguez (93970)}
\maketitle

\section{Procesos}
La aplicación se divide en los siguientes procesos:
\begin{enumerate}
\item Init: llena la cola de autos entrantes.
\item Jefe de estación: recibe los autos entrantes y los asigna a empleados libres. En caso de no haber empleados libres, los despacha.
\item Empleado: asigna autos a surtidores libres. Si no hay surtidores libres, espera a que se libere alguno. Realiza el servicio. Cobra y guarda la recaudación en la caja.
\item Consulta caja registradora: consulta asincrónica sobre el valor de la recaudación en caja.
\end{enumerate}

\section{Comunicación entre procesos}
En el siguiente diagrama, se muestra el esquema de comunicación entre los procesos del sistema. Incluye los datos que intercambian los procesos que se comunican entre sí y los recursos a los que deben acceder de manera concurrente.
\\[1\baselineskip]
\includegraphics[width=\textwidth]{overview}

\section{Problemas conocidos de concurrencia}

\section{Mecanismos de concurrencia a utilizar}

\end{document}
